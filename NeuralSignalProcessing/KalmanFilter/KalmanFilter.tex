% !TEX TS-program = xelatex
% !TEX encoding = UTF-8 Unicode

% This is a simple template for a LaTeX document using the "article" class.
% See "book", "report", "letter" for other types of document.

\documentclass[11pt]{article}

\usepackage[margin=1in]{geometry}
\usepackage{fancyhdr}

\usepackage{parskip}
\usepackage{framed}
\usepackage{amsmath,amsthm,amssymb}

\newtheorem*{conjecture}{Conjecture}

\usepackage{titlesec}

\usepackage[svgnames]{xcolor}

%\usepackage{enumerate}
\usepackage{enumitem}
\newcounter{descriptcount}


%\usepackage{mathpazo}
\usepackage{mathtools}
\usepackage{unicode-math}
\usepackage{empheq}
\usepackage[most]{tcolorbox}
\usepackage{cancel}

\usepackage{caption}

\usepackage{xunicode} %handle unicode
\usepackage{xltxtra} %XeTeX extras
\usepackage{fontspec} %use OTF/TTF fonts

%\newcommand{\lmr}{\fontfamily{lmr}\selectfont} % Latin Modern Roman

\usepackage{tikz}
\usetikzlibrary{calc}
\newcommand{\tikzmark}[1]{\tikz[overlay,remember picture] \node (#1) {};}


%\setmainfont{Myriad Pro} %use this font
%\setmathfont{Adobe Garamond Pro} %use this font for math

\titleformat{\section}
  {\large\bf}{\thesection}{0.25em}{}[\titlerule]
\titlespacing{\section}
  {0pt}{*1.5}{0.25em}

\titleformat{\subsection}
  {\normalfont\bf}{\thesubsection}{0.25em}{}
\titlespacing{\subsection}
  {0pt}{*1}{0.125em}

\renewcommand\thesection{\Alph{section}.}
\renewcommand\thesubsection{\thesection\arabic{subsection}}
\renewcommand\thesubsubsection{(\roman{subsubsection}).}

\renewcommand{\labelitemi}{\textemdash}

\DeclareMathOperator{\dif}{d\!}
%\DeclareMathOperator{\Pr}{P}
\DeclareMathOperator{\E}{E}
\DeclareMathOperator{\Cov}{Cov}
\DeclareMathOperator{\var}{var}
\DeclareMathOperator{\F}{\mathfrak{F}}
\DeclareMathOperator{\Poisson}{Poisson}
\DeclareMathOperator{\Bernoulli}{Bernoulli}
\DeclareMathOperator{\Binomial}{Binomial}
\DeclareMathOperator{\Order}{O}
\DeclareMathOperator{\Uniform}{Uniform}
\DeclareMathOperator*{\argmax}{argmax}
\DeclareMathOperator{\Tr}{Tr}
\DeclareMathOperator{\diag}{diag}
\DeclareMathOperator{\KLsym}{KL}
\DeclarePairedDelimiterX{\klx}[2]{(}{)}{%
  #1\,\delimsize\|\,#2%
}
\newcommand{\KL}{\KLsym\klx}

\newcommand{\logdet}[1]{\log \left| {#1} \right| }

\newcommand{\xb}{\mathbf{x}}
\newcommand{\zb}{\mathbf{z}}
\newcommand{\ub}{\symbf{\mu}}
\newcommand{\Sb}{\symbf{\Sigma}}
\newcommand{\Ab}{\mathbf{A}}
\newcommand{\Qb}{\mathbf{Q}}
\newcommand{\Cb}{\mathbf{C}}
\newcommand{\Rb}{\mathbf{R}}

\newcommand*{\tran}{^{\mkern-1.5mu\mathsf{T}}}
\newcommand*{\invtran}{^{\mkern-1.5mu\mathsf{-T}}}


\newenvironment{propertybox}{%
   \def\FrameCommand{\colorbox{LightSteelBlue}}%
   \MakeFramed{\advance\hsize-\width \FrameRestore}}
 {\endMakeFramed}

\lhead{\textbf{ELEC548}}
\chead{Kalman Filter}
\cfoot{}
\rhead{}
\rfoot{\thepage}
\pagestyle{fancyplain}

%\setlength\parindent{0pt}

\allowdisplaybreaks


\begin{document}
\setmainfont{Myriad Pro} %use this font
%\setmathfont{Latin Modern Math}
%\setmathfont{TG Pagella Math}

\begin{center}
\large
\textbf{ELEC 548} Kalman Filter
\end{center}

\section{Motivation - Continuous Latent Variable Models}
We have implemented latent variable models for Classification, Clustering, and
Dimensionality Reduction. In all of these cases, the data we were modeling were
\textit{static} -- they came from a single snap shot or measurement and our
models did not incorporate the concept of time or continuity.

However, for many machine learning problems, the observed data \textit{vary with
time}. For these problems, we will want our underlying latent variable model to
capture the continuity present in the temporal relationship between data points.
We want models that are \textit{dynamical}.

\section{Linear Dynamical Systems (LDS)}
At time step $t=1, \ldots, T$, let:
\begin{align*}
  \zb_t \in \mathbb{R}^M \quad & \text{be the (latent) ``state'' variable at time $t$} \\
  \xb_t \in \mathbb{R}^D \quad & \text{be the observation at time $t$}
\end{align*}

\subsubsection*{State Model:}
The state model describes how the state evolves over time. Similarly to how we
used a linear Gaussian model for Dimensionality Reduction, pointing out that
both the \textit{linear} and \textit{Gaussian} aspects make this the simplest
interesting model, a Linear Dynamical System with Gaussian ``innovations''
(changes from one time step to the next) is the simplest interesting model for
how the underlying state evolves over time.

\begin{equation}
  \begin{aligned}
    \zb_t \mid \zb_{t-1} &\sim \mathcal{N}(\Ab \zb_{t-1}, \Qb) \\
    \zb_1 &\sim \mathcal{N}(\symbf{\pi}, \mathbf{V})
  \end{aligned}
  \label{eqn:dynamics}
\end{equation}

\subsubsection*{Observation Model:} The observation model describes how the
observed data relates to the state. Similarly to the state mode, the simplest
way of relating the state to our observed data is a linear, Gaussian observation
model.
\begin{equation}
  \xb_t \mid \zb_t \sim \mathcal{N}(\Cb \zb_t, \Rb)
  \label{eqn:observations}
\end{equation}

Thus, the model parameters are $\theta = \{\Ab, \Qb, \symbf{\pi}, \mathbf{V},
\Cb, \Rb$. \textit{Exercise to the reader:} What are the dimensions of each of
these parameters?

\subsubsection*{Markov assumption}
Notice that the state model uses a first-order Markov assumption. Specifically,
we assume that $\zb_{t-1}$, the state at time $t-1$, contains all the relevant
information necessary to predict $\zb_t$, the state at time $t$. This means
that we can write down a simple formula for the joint probability of the state
over all times:
\begin{align*}
  \Pr(\zb_1, \ldots, \zb_T) &= \Pr(\zb_1) \Pr(\zb_2 \mid \zb_1)
      \Pr(\zb_3 \mid \zb_2, \cancel{\zb1}) \ldots
      \Pr(\zb_T \mid \zb_1, \cancel{\zb_2 \ldots \zb_{T-1}})
          \tag{Markov assumption} \\
      &= \Pr(\zb_1) \prod_{t=2}^T \Pr(\zb_t \mid \zb_{t-1})
\end{align*}

\subsection{Training phase}
\begin{tabular}{r l}
  \textbf{Goal:} & Estimate the model parameters $\theta = \{\Ab, \Qb,
  \symbf{\pi},\mathbf{V}, \Cb, \Rb$ from the training data. \\
  \\
  \textit{unsupervised} & If the values of the state variables $\zb_t$
  are \textbf{unknown} during training, use the \textbf{EM algorithm} \\
  \textit{learning} & Maximize $\Pr(\{\xb\} \mid \theta)$ with respect to
  $\theta$. \\
  \\
  \textit{supervised} & If the values of the state variables $\zb_t$
  are \textbf{known} (the simpler case) during training \\
  \textit{learning} & Maximize $\Pr(\{\xb\}, \{\zb\} \mid \theta)$ with
  respect to $\theta$. \\
\end{tabular}

We will consider the simpler, \textit{supervised learning} case, where the
$\zb_t$ are known during training. This makes sense if we are designing, for
example, a real time neural signal processing system in which the thing we wish
to decode (e.g., arm trajectories) can be measured during trianing. In this
context, the latent variable is known during training and unknown during
testing/operation.

\subsubsection{Maxmimum Likelihood Parameters in Supervised Training}
\begin{equation*}
  \Pr(\{\xb\}, \{\zb\} \mid \theta) = \Pr(\zb_1)
    \prod_{t=2}^T \Pr(\zb_t \mid \zb_{t-1})
    \Bigl(\prod_{t=1}^T \Pr(\xb_t \mid \xb_t)\Bigr)
\end{equation*}

Writing down the training data log-likelihood, we have
\begin{align*}
  \mathcal{L}(\theta) &= \log \Pr(\{\xb\}, \{\zb\} \mid \theta) \\
  &= \log \Pr(\zb_1) +
    \sum_{t=2}^T \log \Pr(\zb_t \mid \zb_{t-1}) +
    \sum_{t=1}^T \log \Pr(\xb_t \mid \xb_t) \\
  &= - \frac{M}{2} \log(2 \pi) - \frac{1}{2} \logdet{\mathbf{V}}
       - \frac{1}{2} (\zb_1 - \symbf{\pi})\tran \mathbf{V}^{-1} (\zb_1 - \symbf{\pi}) \\
  &\quad
     + \sum_{t=2}^T \Bigl(-\frac{M}{2} \log(2 \pi) - \frac{1}{2} \logdet{\Qb}
       -\frac{1}{2} (\zb_t - \Ab\zb_{t-1})\tran \Qb^{-1} (\zb_t - \Ab\zb_{t-1})
       \Bigr)\\
&\quad
     + \sum_{t=1}^T \Bigl(-\frac{D}{2} \log(2 \pi) - \frac{1}{2} \logdet{\Rb}
       -\frac{1}{2} (\xb_t - \Cb\zb_t)\tran \Rb^{-1} (\xb_t - \Cb\zb_t) \Bigr)
\end{align*}

Solving for $\Ab$ (which describes the \textit{dynamics} of the state variable):
\begin{align*}
  \frac{\partial \mathcal{L}(\theta)}{\partial \Ab} &=
    \frac{\partial}{\partial \Ab} \Bigl\lbrace \sum_{t=2}^T \Bigl(
     -\frac{1}{2} (\zb_t - \Ab\zb_{t-1})\tran \Qb^{-1} (\zb_t - \Ab\zb_{t-1})
     \Bigr)\Bigr\rbrace \\
  &= -\frac{1}{2} \frac{\partial}{\partial \Ab} \Bigl\lbrace \sum_{t=2}^T \Bigl(
    -\zb_{t-1}\tran\Ab\tran \Qb^{-1} \zb_t
    -\zb\tran \Qb^{-1} \Ab\zb_{t-1}
    +\zb_{t-1}\tran\Ab\tran \Qb^{-1}\Ab\zb_{t-1} \Bigr)\Bigr\rbrace \\
  &= -\frac{1}{2} \frac{\partial}{\partial \Ab} \Bigl\lbrace
    \Tr\left(\Ab\tran \Qb^{-1} \sum_{t=2}^T \zb_t \zb_{t-1}\tran \right)
    +\Tr\left(\Ab \sum_{t=2}^T \zb_t \zb_{t-1}\tran \Qb^{-1} \right)
    +\Tr\left(\Qb^{-1} \Ab \sum_{t=2}^T \zb_t \zb_{t-1}\tran \Ab\tran \right)
    \Bigr\rbrace \\
  &= -\frac{1}{2} \left(-\Qb^{-1} \sum_{t=2}^T \zb_t \zb_{t-1}\tran
    -\Qb^{-1} \sum_{t=2}^T \zb_t \zb_{t-1}\tran
    +\Qb^{-1} \Ab \sum_{t=2}^T \zb_{t-1} \zb_{t-1}\tran
    +\Qb^{-1} \Ab \sum_{t=2}^T \zb_{t-1} \zb_{t-1}\tran \right) \\
  &= 0
\end{align*}
Here, we have made use of the fact that because its a symmetric covariance
matrix, $\Qb\invtran = \Qb^{-1}$, and $\frac{d}{d \mathbf{X}}
\Tr(\mathbf{X} \Ab \tran) = \frac{d}{d \mathbf{X}}
\Tr(\mathbf{X}\tran \Ab) = \Ab$ and $\frac{d}{d \mathbf{X}}
\Tr(\Ab \mathbf{X} \mathbf{B} \mathbf{X}\tran \Cb) =
\Ab\tran\Cb\tran\mathbf{X}\mathbf{B}\tran +
\Cb \Ab \mathbf{X} \mathbf{B}$. Solving, we have
\begin{framed}
  \begin{equation}
    \Ab = \left(\sum_{t=2}^T \zb_t \zb_{t-1} \right)
        \left(\sum_{t=2}^T \zb_{t-1} \zb_{t-1} \right)^{-1}
    \label{eqn:dynamicsSoln}
  \end{equation}
\end{framed}

Solving for $\Qb$ (which describes the variability of the \textit{innovation}, or
change from one time step to the next of the state variable):
\begin{align*}
  \frac{\partial \mathcal{L}(\theta)}{\partial \Qb} &=
    \frac{\partial}{\partial \Qb} \Bigl\lbrace
    -\frac{T-1}{2}\logdet{\Qb} -\frac{1}{2} \Tr\left( \Qb^{-1} \sum_{n=2}^T
    (\zb_t - \Ab \zb_{t-1}) (\zb_t - \Ab \zb_{t-1})\tran \right)\Bigr\rbrace \\
  &= -\frac{T-1}{2} \Qb^{-1} -\frac{1}{2} \left( \Qb^{-1} \sum_{n=2}^T
    (\zb_t - \Ab \zb_{t-1}) (\zb_t - \Ab \zb_{t-1})\tran \Qb^{-1}\right)
  &= 0
\end{align*}
Here, we have additionally used the fact that $\frac{d}{d \mathbf{X}}
\Tr(\mathbf{X}^{-1} \Ab) = -\mathbf{X}^{-1} \Ab \mathbf{X}^{-1}$ and
$frac{d}{d \mathbf{X}} \logdet{\mathbf{X}} = \mathbf{X}^{-1}$. Solving, we have
\begin{framed}
  \begin{equation}
    \Qb = \frac{1}{T-1} \sum_{t=2}^T (\zb_t - \Ab \zb_{t-1})
    (\zb_t - \Ab \zb_{t-1})\tran
    \label{eqn:innovationSoln}
  \end{equation}
\end{framed}
(using the $\Ab$ found above.) Note that these solutions are identical to the
solutions to linear regreissions!

The solutions for $\frac{\partial \mathcal{L}(\theta)}{\partial \Cb} = 0$ and
$\frac{\partial \mathcal{L}(\theta)}{\partial \Rb} = 0$ follow very similar
math and give us the solutions
\begin{framed}
  \begin{equation}
    \Cb = \left(\sum_{t=1}^T \xb_t \zb_t \right)
        \left(\sum_{t=1}^T \zb_t \zb_t \right)^{-1}
    \label{eqn:obsSoln}
  \end{equation}
\end{framed}
and
\begin{framed}
  \begin{equation}
    \Rb = \frac{1}{T} \sum_{t=1}^T (\xb_t - \Rb \zb_t) (\xb_t - \Cb \zb_t)\tran
    \label{eqn:obsNoiseSoln}
  \end{equation}
\end{framed}
(using the solution for $\Cb$ found above).

In these solutions, for simplicity we have considered only one sequence of state
and observation variables. In most scenarios, we would have \textbf{multiple
sequences of training data}, potentially each with a different length ($T$). If
we define $\{\xb\}_n$ and $\{\zb\}_n$ as the $n$-th training sequence
($n = 1, \ldots, N$), then the goal for training would be to find the parameters,
$theta$, which maximize $\prod_{n=1}^N \Pr(\{\xb\}_n, \{\zb\}_n \mid \theta)$.

The maximimum likelihood solutions in the \textbf{multiple sequence} training
case for equations \eqref{eqn:dynamicsSoln} -- \eqref{eqn:obsNoiseSoln} have
the same form, but each summation is over more elements. The solution is
\textit{almost} the same as concatenating all the sequences together, with
the exception that the terms in the dynamics equations, \eqref{eqn:dynamicsSoln}
and \eqref{eqn:innovationSoln} that would involve $\zb_{T,n}$ and $\zb_{1,n+1}$
are not included.

In the case of \textbf{multiple sequences}, we can also solve for the initial
state distribution. $\symbf{\pi}$ and $\mathbf{V}$ are the sample mean and
covariance, respectively, of the $N$ instances of $\zb_1$.

\section{Test Phase / Decoding}
\textbf{Goal:} Compute $\Pr(\zb_t \mid \xb_1, \ldots, \xb_t)$ for $t=1, \ldots, T$.

The variables $\zb_1, \ldots, \zb_T, \xb_1, \ldots, \xb_T$ are jointly
Gaussian, so $\Pr(\zb_t \mid \{\xb\}_1^T)$ is Gaussian (using the notation
$\{\xb\}_1^T = \xb_1, \ldots, \xb_t$). Thus, we only need to find its mean
and covariance.

We can compute $\Pr(\zb_t \mid \{\xb\}_1^T)$ recursively starting at $t=1$.

\begin{align}
  \intertext{\textbf{One-step (forward) prediction}}
  \underbrace{\Pr(\zb_t \mid \{\xb\}_1^{t-1})}_{\tikzmark{a}} &= \int
  \underbrace{\Pr(\zb_t \mid \zb_{t-1})}_{\mathclap{\text{\normalsize state model}}}
  \underbrace{\Pr(\zb_{t-1} \mid \{\xb\}_1^{t-1})}_{\tikzmark{c}}
  \label{eqn:forward}\\
  \intertext{\textbf{Measurement update}}
  \overbrace{\Pr(\zb_t \mid \{\xb\}_1^t)}^{\tikzmark{d}} &= \frac{
  \overbrace{\Pr(\zb_t \mid \zb_t)}^{\mathclap{\text{\normalsize obs. model}}}
  \overbrace{\Pr(\zb_t \mid \{\xb\}_1^{t-1})}^{\tikzmark{b}}}
  {\Pr(\xb_t \mid \{\xb\}_1^{t-1})}
  \label{eqn:update}
  \begin{tikzpicture}[overlay,remember picture,very thick]
    \draw (a.center) edge[out=270, in=90, ->,red] (b.north);
    \draw (d.north) edge[out=90, in=270, ->,blue]  (c.center);
  \end{tikzpicture}
\end{align}



Note that equations \eqref{eqn:forward} and \eqref{eqn:update} are always a
valid way of describing a dynamical system with Markovian properties. When
we specify ``linear'' and ``Gaussian'', then each component is Gaussian, and
all that we need to calculate are the mean and covariance. Define
\begin{align*}
  \ub_t^{\tau} &= \E(\zb_t \mid \{\xb\}_1^{\tau}) \\
  \Sb_t^{\tau} &= \Cov(\zb_t \mid \{\xb\}_1^{\tau})
\end{align*}

\subsubsection*{One-step prediction}
\begin{equation*}
  \zb_t \mid \{\xb\}_1^{t-1} \sim \mathcal{N}(\ub_t^{t-1}, \Sb_t^{t-1})
\end{equation*}
So we need to find $\ub_t^{t-1}$ and $\Sb_t^{t-1}$. We can equivalently write
\eqref{eqn:dynamics} as
\begin{equation*}
  \zb_t = \Ab \zb_{t-1} + \mathbf{v}_t, \quad \mathbf{v}_t \sim
    \mathcal{N}(\mathbf{0}, \Qb)
\end{equation*}
Thus,
\begin{align*}
  \ub_t^{t-1} &= \E(\zb_t \mid \{\xb\}_1^{t-1}) \\
  &= \Ab  \E(\zb_{t-1} \mid \{\xb\}_1^{t}) +
    \cancel{\E(\mathbf{v}_t \mid \{\xb\}_1^{t-1})}
\end{align*}
\begin{framed}
  \begin{equation}
    \ub_t^{t-1} = \Ab \ub_{t-1}^{t-1}
  \end{equation}
\end{framed}
and
\begin{align*}
  \Sb_t^{t-1} &= \Cov(\zb_t \mid \{\xb\}_1^{t-1}) \\
  &= \E((\Ab \zb_{t-1} + \mathbf{v}_t) (\Ab \zb_{t-1} + \mathbf{v}_t)\tran) -
     \E(\zb_t \mid \{\xb\}_1^{t-1}) \E(\zb_t \mid \{\xb\}_1^{t-1})\tran \\
  &= \E(\Ab \zb_{t-1} \zb_{t-1}\tran \Ab\tran +
      \cancel{\mathbf{v}_t \zb_{t-1}\tran \Ab\tran} +
      \cancel{\Ab \zb_{t-1} \mathbf{v}_t\tran} +
      \mathbf{v}_t \mathbf{v}_t)\tran) -
      \Ab \ub_{t-1}^{t-1} (\ub_{t-1}^{t-1})\tran \Ab\tran \\
  &= \Ab  \Cov(\zb_{t-1} \mid \{\xb\}_1^{t}) \Ab\tran +
    \Cov(\mathbf{v}_t \mid \{\xb\}_1^{t-1})
\end{align*}
\begin{framed}
  \begin{equation}
    \Sb_t^{t-1} = \Ab \Sb_{t-1}^{t-1} \Ab\tran + \Qb
  \end{equation}
\end{framed}

\subsubsection*{Measurement update}
\begin{equation*}
  \zb_t \mid \{\xb\}_1^{t} \sim \mathcal{N}(\ub_t^t, \Sb_t^t)
\end{equation*}
So we need to find $\ub_t^t$ and $\Sb_t^t$. Notice that \eqref{eqn:update} is
just Bayes rule for $\zb_t$ given $\xb_t$ with everything conditioned on
$\{\xb\}_1^{t-1}$. So inspired by what we learned in our analysis of Dimensionality
Reduction (P-PCA) about conditional Gaussian distributions from the joint
distribution, lets start by finding the joint distribution
$\Pr(\zb_t,\xb_t \mid \{\xb\}_1^{t-1})$.

We will start by noting the equivalent form of
\eqref{eqn:observations} is
\begin{equation*}
  \xb_t = \Cb \zb_t + \mathbf{w}_t, \quad \mathbf{w}_t \sim
    \mathcal{N}(\mathbf{0}, \Rb)
\end{equation*}
We need $\E(\xb_t\mid\{\xb\}_1^{t-1})$, $\Cov(\xb_t\mid\{\xb\}_1^{t-1})$, and
the cross covariance $\E(\xb_t\,\zb_t\tran \mid\{\xb\}_1^{t-1})$.
\begin{align}
  \E(\xb_t\mid\{\xb\}_1^{t-1}) &= \Cb \E(\zb_t\mid\{\xb\}_1^{t-1}) +
    \cancel{\E(\mathbf{w}_t \mid\{\xb\}_1^{t-1})} \nonumber \\
    &= \Cb \ub_t^{t-1}
\end{align}

\begin{align}
  \Cov(\xb_t\mid\{\xb\}_1^{t-1}) &= \Cb \Cov(\zb_t\mid\{\xb\}_1^{t-1}) \Cb\tran +
    \Cov(\mathbf{w}_t \mid\{\xb\}_1^{t-1}) + \cancel{\E(\text{cross terms})}\nonumber \\
    &= \Cb \Sb_t^{t-1} \Cb\tran + \Rb
\end{align}

\begin{align}
  &\E(\xb_t \, \zb_t\tran \mid\{\xb\}_1^{t-1}) - \E(\xb_t\mid\{\xb\}_1^{t-1})
      \E(\zb_t\mid\{\xb\}_1^{t-1})\tran \\
   &\qquad\qquad = \Cb \E(\zb_t \,\zb_t\tran \mid\{\xb\}_1^{t-1}) +
       \cancel{\E(\mathbf{v}_t \, \zb_t\tran \mid\{\xb\}_1^{t-1})} -
       \Cb \ub_t^{t-1} {\ub_t^{t-1}}\tran \\
  &\qquad\qquad = \Cb \Sb_t^{t-1}
\end{align}

Filling in, we have

\begin{equation*}
  \begin{bmatrix}\xb_t \\ \zb_t\end{bmatrix} \mid \{\xb\}_1^{t-1} \sim
    \mathcal{N}\left( \begin{bmatrix}\Cb \ub_t^{t-1} \\ \ub_t^{t-1}\end{bmatrix},
      \begin{bmatrix}\Cb \Sb_t^{t-1} \Cb\tran + \Rb & \Cb \Sb_t^{t-1} \\
                     \Sb_t^{t-1} \Cb\tran &
                     \Sb_t^{t-1}\end{bmatrix}
    \right)
\end{equation*}

Now, applying the rule for conditioning in jointly Gaussian random variables
(see Dimensionality Reduction notes), we have

\begin{align*}
  \ub_t^t &= \E(\zb_t \mid \xb_t, \{\xb\}_1^{t-1}) \\
      &= \ub_t^{t-1} + \underbrace{\Sb_t^{t-1} \Cb\tran
          (\Cb \Sb_t^{t-1} \Cb\tran + \Rb)^{-1}}
          (\xb_t - \Cb \ub_t^{t-1})
\end{align*}
The highlighted term, $\Sb_t^{t-1} \Cb\tran (\Cb \Sb_t^{t-1} \Cb\tran + \Rb)^{-1}$,
is often defined as a single matrix, $\mathbf{K}_t$, or the ``Kalman gain''.
Rewriting, we have
\begin{framed}
  \begin{equation}
    \ub_t^t = \ub_t^{t-1} + \mathbf{K}_t (\xb_t - \Cb \ub_t^{t-1})
  \end{equation}
\end{framed}

\begin{align*}
  \Sb_t^t &= \Cov(\zb_t \mid \xb_t, \{\xb\}_1^{t-1}) \\
      &= \Sb_t^{t-1} + \Sb_t^{t-1} \Cb\tran
      (\Cb \Sb_t^{t-1} \Cb\tran + \Rb)^{-1} \Cb \Sb_t^{t-1}
\end{align*}

\begin{framed}
  \begin{equation}
    \Sb_t^t = \Sb_t^{t-1} + \mathbf{K}_t \Cb \Sb_t^{t-1}
  \end{equation}
\end{framed}

So we can now restate the recursive computation in terms of the means and
covariances:

\begin{equation}
\begin{aligned}
  \textbf{One-step (forward) prediction} \\
  \Pr(\zb_t \mid \{\xb\}_1^{t-1}) &= \int
  \Pr(\zb_t \mid \zb_{t-1})
  \Pr(\zb_{t-1} \mid \{\xb\}_1^{t-1}) \\
  \ub_t^{t-1} &= \Ab \ub_{t-1}^{t-1}, \\
  \Sb_t^{t-1} &= \Ab \Sb_{t-1}^{t-1} \Ab\tran + \Qb \\
  \\
  \textbf{Measurement update} \\
  \Pr(\zb_t \mid \{\xb\}_1^t) &= \frac{\Pr(\zb_t \mid \zb_t)
    \Pr(\zb_t \mid \{\xb\}_1^{t-1})}
  {\Pr(\xb_t \mid \{\xb\}_1^{t-1})} \\
  \ub_t^t &= \ub_t^{t-1} + \mathbf{K}_t (\xb_t - \Cb \ub_t^{t-1}),\\
  \Sb_t^t &= \Sb_t^{t-1} + \mathbf{K}_t \Cb \Sb_t^{t-1}, \\
  \text{where } \mathbf{K}_t &= \Sb_t^{t-1} \Cb\tran (\Cb \Sb_t^{t-1} \Cb\tran + \Rb)^{-1}  \\
\end{aligned}
\label{eqn:recursion}
\end{equation}

Using these recursions, we can obtain $\ub_t^t$ and $\Sb_t^t$ for $t = 1, \ldots
T$. What do they mean?
\begin{itemize}
  \item $\ub_t^t$ is the estimate of the state at time $t$
  \item $\Sb_t^t$ is our uncertainty around that estimate at time $t$
\end{itemize}

\textit{How do we initialize the recursion?} We initialize with $\ub_1^0 =
\symbf{\pi}$ and $\Sb_1^0 = \mathbf{V}$.

\section{Unsupervised learning?}
How do we train a linear Gaussian dynamical system model if we don't have
training data? The unsupervised learning solution is described in Chapter 12 of
\textit{PRML}. Briefly, the EM algorithm iterates between state estimation via
\eqref{eqn:recursion} and ML parameter estimation using sufficient statistics
calculated from those state estimates.

\end{document}
